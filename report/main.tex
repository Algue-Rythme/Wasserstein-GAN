\documentclass[a4paper]{report}

%% Language and font encodings
\usepackage[english]{babel}
\usepackage[utf8x]{inputenc}
\usepackage[T1]{fontenc}
\usepackage{pdflscape}
\usepackage{xspace}
\usepackage[table,xcdraw]{xcolor}
\usepackage{multirow}
\usepackage{float}
\usepackage{titlesec}
\usepackage{lipsum}

%% Sets page size and margins
\usepackage[a4paper,top=3cm,bottom=2cm,left=3cm,right=3cm,marginparwidth=1.75cm]{geometry}

%% Useful packages
\usepackage{amsmath}
\usepackage{amssymb}
\usepackage{amsfonts}
\usepackage{amsthm}
\usepackage{graphicx}
\usepackage[colorinlistoftodos]{todonotes}
\usepackage[colorlinks=true, allcolors=blue]{hyperref}
\usepackage[linesnumbered, ruled, vlined]{algorithm2e}
\usepackage{algorithmic}

\titleformat{\chapter}
  {\normalfont\LARGE\bfseries}{\thechapter}{1em}{}
\titlespacing*{\chapter}{0pt}{3.5ex plus 1ex minus .2ex}{2.3ex plus .2ex}


\begin{document}

\begin{titlepage}
\begin{center}
 {\Huge\bfseries Wassertein Generative Adversarial Networks (WGAN)\\}
 \vspace{2cm}
 {\Large \bfseries Machine Learning Project \\}
 \vspace{2cm}
 {\large ENS de Lyon, Spring 2018 \\}
 \vspace{2cm}
{\Large \urlstyle{same} \color{black}
	\href{mailto:louis.bethune@ens-lyon.fr}{louis.bethune@ens-lyon.fr}\\
    \vspace{0.2cm}
	\href{mailto:guillaume.coiffier@ens-lyon.fr}{guillaume.coiffier@ens-lyon.fr}\\
	\vspace{0.2cm}
	\href{mailto:leonard.assouline@ens-lyon.fr}{leonard.assouline@ens-lyon.fr}\\
}
\vspace{2cm}
\vfill
\end{center}
\end{titlepage}




\setcounter{tocdepth}{1}
\tableofcontents





\chapter*{Introduction}

\lipsum

\chapter{Theory of Generative Adversarial Networks}

\section{Generative Adversarial Networks}
\lipsum[1]

\section{The problem of learning}
\lipsum[1]

\section{Wassertein GAN}
\lipsum[1]

\chapter{Applications}
\lipsum


\nocite{*}
\bibliographystyle{unsrt}
\bibliography{biblio}
\end{document}
